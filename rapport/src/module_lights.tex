\section{Lumières}
L'ensemble des sources de lumières doivent respecter l'interface de la \tsl{fig.
\ref{fig:ILights}}. Concrètement, chaque caméra doit pouvoir, à partir des
coordonnées de l'espace écran (coordonnées UV), fournir l'ensemble des rayons
nécessaire au dessin de la scène.

\begin{figure}[h]
  \inputcode{c++}{../../code/src/Lights/Light.hpp}
  \caption{Code de l'interface commune à toutes les
  sources de lumières\label{fig:ILights}}
\end{figure}

\subsection{Spécialisations}
\paragraph{[Implémentée] Lumière directionnelle} Cette source éclaire tout selon une
direction passée en paramètre.

\paragraph{[Implémentée] Lumière ponctuelle} Cette source, définit grâce à sa position,
rayonne autour d'elle comme une sphère lumineuse.

\paragraph{[Implémentée] Lumière surfacique plane} Cette source a la particularité d'avoir une
composante surfacique ce qui permet de générer des ombres douces (\cf
\tsl{section \ref{sec:softshadows}}).

\paragraph{[Non implémentée] Cone de lumière}

\paragraph{[Non implémentée] Autre lumières surfaciques} Il est possible de
rajouter tout les types de lumières surfaciques comme des sphères lumineuse et
même des mesh lumineux. 

\subsection{Diagramme de classe}
\begin{figure}[h]
\begin{center}
  \includegraphics[width=\textwidth]{../../architecture/Lights}
  \caption{Diagramme de classe du module Light\label{fig:CDLights}}
\end{center}
\end{figure}
