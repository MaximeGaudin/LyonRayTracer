\section{Lumières}
L'ensemble des sources de lumières doivent respecter l'interface de la
\tsl{fig.\ref{fig:ILights}}. 

\begin{figure}[h]
  \inputcode{c++}{../../code/src/Lights/Light.hpp}
  \caption{Code de l'interface commune à toutes les
  sources de lumières\label{fig:ILights}}
\end{figure}

\subsection{Spécialisations}
\paragraph{[Implémentée] Lumière directionnelle} Cette source éclaire tout
selon une direction passée en paramètre. Elle simule une source très lointaine
dont tous les rayons sont parallèles.

\paragraph{[Implémentée] Lumière ponctuelle} Cette source définie par sa
position, rayonne autour d'elle comme une point lumineuse.

\paragraph{[Implémentée] Lumière surfacique plane} Cette source a la
particularité d'avoir une composante surfacique qui permet de générer des
ombres douces (\cf \tsl{section \ref{sec:softshadows}}).

\paragraph{[Non implémentée] Cone de lumière}

\paragraph{[Non implémentée] Autre lumières surfaciques} Il est possible de
rajouter tout les types de lumières surfaciques comme des sphères lumineuses et
même des \eng{mesh}\ lumineux. 

\begin{figure}[h]
  \hbox to \textwidth
  {\hss\includegraphics[width=1.3\textwidth]{../../architecture/Lights}\hss}
  \caption{Diagramme de classe du module Light\label{fig:CDLights}}
\end{figure}
