\subsection{Non-objectifs}
Comme dans tout projet, celui-ci possède des non-objectifs, \ie des objectifs
que j'ai volontairement choisi de ne pas atteindre, ou en tout cas de ne pas
viser.\\

D'abord, concernant la vitesse d'exécution. Les \eng{raytracers} commerciaux
sont de réelles prouesses d'ingénierie et d'optimisation et j'ai considéré que
l'optimisation de mon programme sortait du contexte d'un projet personnel.\\

Il ne s'agissait pas non plus d'en faire un produit fini mais plutôt un
\tsl{proof of concept}. Il reste donc de nombreuses améliorations tant au
niveaux des fonctionnalités existantes qu'au niveau de celles à ajouter.
