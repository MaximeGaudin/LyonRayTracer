\chapter{Manuel utilisateur}
\section{Compilation}
Le \eng{pipeline} de compilation est tout ce qu'il y a de plus classique :
\begin{center}
\begin{verbatim}
./configure
make
\end{verbatim}
\end{center}

\subsection{Dépendances}
Afin de pouvoir compiler ce programme, le script \ttt{configure} va s'assurer
que vous posséder :
\begin{itemize}
  \item boost @1.47.0, Revision 2 $(>=)$
  \item lib3ds @20080909 $(>=)$
  \item libpng @1.4.8 $(==)$
\end{itemize}

\section{Licence}
C'est la première fois que j'ai à faire au problème des license et avec
seulement 3 bibliothèque, cela semble déjà être un casse tête. Je crois
cependant que la license la plus restrictive est la \tsl{Boost licence version
1.0} et que par conséquent, ce programme doit être distribué sous cette
license.

\section{Utilisation}
L'utilisation du programme est très simple :
\begin{center}
\begin{verbatim}
./lrt SCENE.lrt [fichierDeSortie.png]
\end{verbatim}
\end{center}

\remark{Par défaut, le fichier de sortie est \ttt{result.png}.}
