\section{Journalisation}
Pour un programme d'une telle ampleur, il est important de pouvoir retracer le
l'exécution sans avoir à utiliser un debugger. C'est pourquoi j'ai choisi de
me tourner vers une gestion centralisée et systématique des logs.

\subsection{Exemple :} Rendu de la scène \ttt{DOF.lrt} 
\begin{center}
\begin{verbatim}
[Core] - Builders discovering...
[SceneReader] - 'material' builder added.
[SceneReader] - 'perspective' builder added.
[SceneReader] - 'perspectiveDOF' builder added.
[SceneReader] - 'point' builder added.
[SceneReader] - 'area' builder added.
[SceneReader] - 'sphere' builder added.
[SceneReader] - 'plane' builder added.
[SceneReader] - 'mesh' builder added.
[Core] - Building scene...
[SceneReader] - Reading scene : scenes/DOF.lrt ...
[SceneReader] - Image...
[Image] - Building new output frame (600, 600)... OK
[SceneReader] - Camera...
[PerspectiveDOF] - New Camera : Eye - ( 0 ,0 ,8 ), LookAt - ( 0 ,0 ,0 ),
Direction ( 0 ,0 ,-1 )
[SceneReader] - Materials...
[MaterialBuilder] - New material with diffuse = [ R=1, G=0, B=0 ].
[MaterialBuilder] - New material with diffuse = [ R=1, G=1, B=0 ].
[MaterialBuilder] - New material with diffuse = [ R=0, G=1, B=0 ].
[MaterialBuilder] - New material with diffuse = [ R=0, G=0, B=1 ].
[SceneReader] - Lights...
[SphereBuilder] - New point light at ( -20 ,20 ,20 )
[SceneReader] - Geometry...
[SphereBuilder] - New sphere at ( 0 ,0 ,0 ) (radius = 1)
[SphereBuilder] - New sphere at ( -1 ,0 ,4 ) (radius = 1)
[SphereBuilder] - New sphere at ( 2 ,0 ,-4 ) (radius = 1)
[PlaneBuilder] - New plane at ( 0 ,-1.2 ,0 ) (normal= ( 0 ,1 ,0 ))
[Core] - Rendering...

0%...........................................................
10%...........................................................
20%...........................................................
...
100%..........................................................
[Core] - Saving...
[Core] - Cleanup...
\end{verbatim}
\end{center}

\subsection{Améliorations}
\begin{description}
  \item [Gestion des niveaux d'alerte] Alors que certaines informations
    peuvent être utiles à l'utilisateur, d'autres sont purement fonctionnelles
    et permettent un débugage plus rapide. Il serait donc judicieux de pouvoir
    affecter des niveaux aux événements de journalisation dans le but de
    pouvoir les filtrer.
\end{description}

\subsection{Bug connu} 
Aucun.
