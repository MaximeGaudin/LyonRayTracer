\section{Gestionnaire de scènes\label{scenemanager}}
Un des élément principaux, il permet grâce à une description textuelle, de
générer la représentation mémoire de la scène. C'est par conséquent
l'interface de communication entre l'utilisateur et le client.

La démocratisation et la facilité du langage XML m'ont poussé à le choisir
comme métalangage de description. De plus, la disponibilité d'outils pour le
\eng{parser}\ rend son utilisation beaucoup plus simple\footnote{Dans mon cas,
la \tsl{parsing}\ est assuré par \tsl{Boost}\ ce qui le rend très solide.}.

Enfin, il augmente l'interopérabilité avec d'autre logiciel comme les
modeleurs par exemple.

\subsection{Exemple} 
La code suivant est un exemple simple de scène mettant en jeu 3 sphères, 1 plan
et une caméra avec gestion de la profondeur de champ :

\inputcode{xml}{../../code/scenes/DOF.lrt}

Comme ci-dessus, une description de scène est composée de 4 blocs :
\begin{enumerate}
  \item Un nœud \ttt{camera} : C'est ici qu'est déclaré l'\tsl{unique}\ caméra
  de la scène.
  \item Un nœud \ttt{materials} : C'est ici que doivent être déclaré l'ensemble
  des matériaux identifiés par une chaine de caractère \tsl{unique}.
  \item Un nœud \ttt{lights}.
  \item Un nœud \ttt{geometries}.
\end{enumerate}
\remark{L'ordre n'est pas imposé.}\vspace*{1em}

Comme précisé dans l'architecture, c'est chaque \eng{builder}\ qui décrit les
paramètres dont il a besoin pour fonctionner. Certains sont obligatoires,
d'autres optionnels.

\subsection{Améliorations}
\begin{description}
  \item [Gestion de la casse] Pour l'instant, la détection des paramètres est
    \tsl{case-sensitive}, c'est inutile et source d'erreurs difficiles à
    débusquer.
\item [Auto-description] Chaque \tsl{builder}\ devrait pouvoir se décrire via
  la ligne de commande en expliquant quels sont les paramètres dont il a
  besoin et leurs domaines de définitions.
\end{description}

Ces deux améliorations sont relativement simples à implémenter et c'est
principalement le développement d'autres fonctionnalité et le manque de temps
qui m'ont empêcher de les ajouter au programme. 

Il faut tout de même noter que la seconde amélioration n'avait pas été prévu
et qu'il faudra par conséquent changer l'interface des \tsl{builder}, ce qui impacte
l'ensemble de code existant.  C'est donc un oubli très gênant de ma part.

\subsection{Bug connu}
Aucun.
