\subsection{Objectifs du projet}
Les objectifs du projet sont nombreux, aussi bien sur le plan de
l'architecture logiciel, que de la gestion de projet ou encore u
développement.

\paragraph{Architecture logiciel}
Un des buts principal du projet (si ce n'est le but principal du projet) est
de créer un programme ouvert à l'extension et pouvant servir de base à la
création d'un \raytracing plus évolué, ou plus rapide, ou plus facile à
utiliser, etc.

\paragraph{Portabilité}
Afin d'être le plus portable possible, j'ai décidé d'utiliser un mélange
d'outils standards comme :
\begin{itemize}
  \item C/C++ : Compilable sur presque toutes les plateformes.
  \item Les autotools\footnote{\url{http://sources.redhat.com/autobook/}} :
  Système de build standard et ne nécessitant aucun autre programme que make
  et bash.
  \item Boost : La bibliothèque C++ de référence.
  \item libpng/libjpg : Pour l'import texture et l'export des résultats.
  \item lib3ds : Pour l'importation des modèles 3D.
\end{itemize}

\paragraph{Gestion de projet}
J'ai réalisé ce projet seul du 24 Septembre au 1 Décembre en utilisant un
style de développement \gls{XP}\footnote{\url{http://www.extremeprogramming.org/}}.
Le gestionnaire de version utilisé est git et le projet ainsi que son
\href{http://digitalguru.github.com/LyonRayTracer}{site web} sont hébergés sur
GitHub.

\paragraph{Implémentation}
Il était pour moi primordial de respecter tout au long de ce projet des
\eng{guidelines} afin d'assurer une cohérence de tout le code (de plusieurs
miliers de lignes tout de même). La documentation fait aussi partie intégrante
du processus de développement et la génération de celle-ci fait partie du
\eng{pipeline} de compilation.

\paragraph{Fonctionnalité du ray tracer}
Reprenons les objectifs fixés dans le sujet :
\begin{enumerate}
  \item Un moteur de \raytracing ``classique''.\textcolor[rgb]{0.0, 0.9, 0.2}{\checkmark}
  \item La gestion de primitives (sphères, plan, tores, et en général toutes
    les figures ayant une représentation paramétrique).\textcolor[rgb]{0.0, 0.9, 0.2}{\checkmark}
  \item La gestion de plusieurs types de caméras : Orthographique,
    perspective, \etc\textcolor[rgb]{0.0, 0.9, 0.2}{\checkmark}
  \item La gestion de plusieurs types de lumières : Point, plan, sphérique,
    \etc.\textcolor[rgb]{0.0, 0.9, 0.2}{\checkmark}
  \item La gestion d'au moins un format de représentation polygonale.\textcolor[rgb]{0.0, 0.9, 0.2}{\checkmark}
  \item La mise en place des structures accélératrices.\textcolor[rgb]{0.0, 0.9, 0.2}{\checkmark}
\end{enumerate}

\newpar J'ai aussi pris le temps d'implémenter :
\begin{enumerate}
  \item Un chargeur de scènes XML.
  \item La lecture et l'écriture d'images au format PNG ou JPG (avec la
  possibilité d'ajouter d'autre formats).
  \item La gestion des textures.
  \item Le \gls{supersampling}.
\end{enumerate}


