\chapter{Objectifs}
Les objectifs du projet sont nombreux, aussi bien sur le plan de
l'architecture logiciel, que de la gestion de projet, de la rédaction ou
encore du développement.

\paragraph{Architecture logiciel -} Un des principaux buts du projet (si ce
n'est le but principal du projet) est de créer un programme ouvert à
l'extension et pouvant servir de base à la création d'un \eng{raytracer}\ plus
évolué, plus rapide, ou plus facile à utiliser.

L'architecture du programme est défini à la section \ref{sec:architecture}.

\paragraph{Portabilité -} Afin d'être le plus portable possible, j'ai décidé
d'utiliser un ensemble d'outils standards et libres :
\begin{itemize}
  \item \tsl{Le langage C++ :}\ \eng{Open-Spec}\ et compilable sur presque
    toutes les plateformes.
  \item \tsl{Les autotools}\footnote{\url{http://sources.redhat.com/autobook/}
    :}\ Système de build standard et ne nécessitant aucun autre programme que
    make et bash. De plus, il assure une portabilité maximale en vérifiant la
    présence de toutes les bibliothèques nécessaire à la compilation et au bon
    fonctionnement du programme.
  \item \tsl{Boost :}\ La bibliothèque C++ de référence.
  \item \tsl{libpng/libjpg :}\ Pour l'import texture et l'export des résultats.
  \item \tsl{lib3ds :}\ Pour l'importation des modèles 3D.
  \item \tsl{\LaTeX{} :}\ Pour la rédaction du rapport.\\

  \item \tsl{Tim Horton :}\ Pour le café et les beignets.
\end{itemize}

\paragraph{Gestion de projet -} J'ai réalisé ce projet seul du 24 Septembre au
1 Décembre en utilisant un style de développement
\gls{XP}\footnote{\url{http://www.extremeprogramming.org/}}.  Le gestionnaire
de version utilisé est \tsl{git}\ et le projet ainsi que son
\href{http://digitalguru.github.com/LyonRayTracer}{site web} sont hébergés sur
\href{http://github.com}{GitHub}.

\remark{Parce que la charge de travail aurait été trop grande, j'ai décidé de
ne pas rédiger les documents de références du type \tsl{Dossier
d'initialiation}, \tsl{Dossier de jalonnage}, \tsl{Planning prévisionnel},
\etc J'ai cependant pris soin de rédiger une documentation exhaustive de
chacune des classes du projet.}
\vspace*{1em}

\paragraph{Implémentation -} Il était pour moi primordial de respecter tout au
long de ce projet des \eng{guidelines}\ afin d'assurer une cohérence de tout
le code (plusieurs milliers de lignes tout de même). J'ai pour cela suivit
principalement le guide de style définit par Herb \tsc{Sutter} dans
\cite{Sutter05}. Je me suis aussi inspiré de \cite{Meyers94}.\\

Enfin, la documentation fait aussi partie intégrante du processus de
développement et la génération de celle-ci fait partie du \eng{pipeline} de
compilation.

\paragraph{Fonctionnalité du ray tracer -} Reprenons les objectifs fixés dans
le sujet :
\begin{enumerate}
  \item \tickmark{} Un moteur de \raytracing\ ``classique''.
  \item \tickmark{} La gestion de primitives (sphères, plan, boite,
    triangles).
  \item \tickmark{} La gestion de plusieurs types de caméras : Orthographique,
    perspective, \etc
  \item \tickmark{} La gestion de plusieurs types de lumières : Point, plane,
    directionnelle, \etc
  \item \tickmark{} La gestion d'au moins un format de représentation
    polygonale.
  \item \tickmark{} La mise en place des structures accélératrices.
  \item \tackmark{} La gestion du \eng{photon mapping} : Comme je l'avais
    pressentie dans mon sujet, le temps alloué au projet était insuffisant
    pour l'implémentation d'une telle fonctionnalité.\\
\end{enumerate}

J'ai cependant pris le temps d'implémenter en plus :
\begin{enumerate}
  \item Un chargeur de scènes XML.
  \item La lecture et l'écriture d'images au format PNG ou JPG (avec la
    possibilité d'ajouter d'autre formats).
  \item La gestion des textures.
  \item Le \gls{supersampling}.
  \item La gestion des ombres douces.
  \item La gestion de la profondeur de champ.
\end{enumerate}

\subsection{Non-objectifs}
Comme dans tout projet, celui-ci possède des non-objectifs, \ie des objectifs
que j'ai volontairement choisi de ne pas atteindre, ou en tout cas de ne pas
viser.\\

D'abord, concernant la vitesse d'exécution. Les \eng{raytracers} commerciaux
sont de réelles prouesses d'ingénierie et d'optimisation et j'ai considéré que
l'optimisation de mon programme sortait du contexte d'un projet personnel.\\

Il ne s'agissait pas non plus d'en faire un produit fini mais plutôt un
\tsl{proof of concept}. Il reste donc de nombreuses améliorations tant au
niveaux des fonctionnalités existantes qu'au niveau de celles à ajouter.
 % CHECKED
