\section{Camera}
L'ensemble des caméras doivent respecter l'interface de la \tsl{fig.
\ref{fig:ICamera}}. Concrètement, chaque caméra doit pouvoir, à partir des
coordonnées de l'espace écran (coordonnées UV), fournir l'ensemble des rayons
nécessaires au dessin de la scène.

\begin{figure}[h]
  \inputcode{c++}{../../code/src/Cameras/Camera.hpp}
  \caption{Code de l'interface commune à toutes les
  caméras\label{fig:ICamera}}
\end{figure}

\subsection{Spécialisations}
\paragraph{[Implémentée] Caméra perspective} Une caméra simple avec gestion de
la perspective.

\paragraph{[Implémentée] Caméra perspective DOF} Une caméra avec
gestion de la perspective et rendu de l'effet de profondeur de champs (\cf
\tsl{section \ref{sec:dof}}).

\paragraph{[Non implémentée] Caméra orthographique} Caméra avec projection
orthographique (utiles pour les rendu de pièce en 3D et le métrage).

\paragraph{[Non implémentée] Caméra fish eye} Cette caméra permet d'utiliser
des matrices de projection grand angle.

\begin{figure}[h]
\begin{center}
  \noindent\makebox[\textwidth]{%
    \includegraphics[width=1.5\textwidth]{../../architecture/Cameras}}
  \caption{Diagramme de classe du module Caméra\label{fig:CDCamera}}
\end{center}
\end{figure}
