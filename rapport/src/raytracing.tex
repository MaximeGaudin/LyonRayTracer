\section{Le \raytracing ?}
Le \raytracing{}\ est une technique de rendu qui, à la manière des réseaux de
neurones, tente de s'approcher au plus prêt de la réalité physique des
phénomènes dans l'espoir d'obtenir un résultat. 

À défaut de décrire le \tsl{pipeline}\ complet d'un \tsl{ray tracer}\ (ce
n'est pas l'objectif du rapport), il m'a tout de même semblé important
d'écrire une brève introduction sur ce qui pourrait bien être \tsl{la
principale méthode de rendu dans un futur (très) proche.}\footnote{À ce sujet,
je vous invite à consulter les travaux de
\href{http://developer.nvidia.com/optix}{NVidia} et
\href{http://www.tgdaily.com/business-and-law-features/36606-inside-intels-ray-tracing-research}{Intel}
sur le ray tracing.}

\subsection{Le modèle de propagation de la lumière}
Comme son nom l'indique, l'idée du \raytracing{}\ est de calculer le chemin que
parcourent les rayons lumineux au travers d'une scène virtuelle.

Dans la réalité, l'œil humain perçoit les objets grâce à la lumière qu'ils
réfléchissent. Nous pourrions donc imaginer calculer l'ensemble des
interactions lumineuses d'une scène en partant de toutes les sources de
lumière. 

Un tel modèle supposerai cependant une puissance de calcul infinie et pour que
les calculs restent raisonnables, les simplifications suivantes ont dû être
faites : 
\begin{enumerate}
  \item \tbf{La lumière se propage en ligne droite :} Nous abandonnons donc le
    modèle ondulatoire de la lumière pour nous restreindre à l'optique
    géométrique.\footnote{Nous verrons tout de même dans la \tsl{sec.
    \ref{sec:state_of_the_art}}\ qu'il est possible d'intégrer des phénomènes
    ondulatoires au rendu.}

  \item \tbf{La lumière que nous recevons est la même que si la source était notre
    œil :} En effet, plutôt que de lancer la majorité des rayons dans le vide
    (\ie des rayons qui ne parviendraient pas jusqu'à notre œil), il est plus
    judicieux de lancer les rayons depuis la caméra jusqu'à notre scène --- 
    \tsl{C'est le Principe du retour inverse de la lumière de Fermat.}

  \item \tbf{Le nombre de rebonds que fait la lumière est fini :} En optique
    géométrique, il n'y a pas de perte d'énergie et considérer tous les rebonds
    mènerai à des récursions infinies.\footnote{Comme pour les phénomène
    ondulatoire, nous verrons qu'il est possible de raffiner le modèle pour
    intégrer des pertes d'énergies (\cf \tsl{sec. \ref{sec:radiosity}}).}
\end{enumerate}

\subsection{L'algorithme simplifié :}
Cette courte introduction nous amène à l'algorithme de rendu suivant :
\begin{algorithm}[H]
\caption{getPixel}
\begin{algorithmic}
\REQUIRE X, l'abscisse du point dont on veux la couleur.
\REQUIRE Y, l'ordonnée du point dont on veux la couleur.
\REQUIRE L, le niveau de récursion, \ie le nombre de fois que le rayon a été
réfléchi ou réfracté.
\ENSURE La couleur du pixel demandé.

\STATE FinalColor $\leftarrow$ BLACK
\IF {L $\gt$ MAX\_RECURSION\_LEVEL}
  \RETURN FinalColor
\ENDIF
\STATE
\STATE Ray $\leftarrow$ Camera.getRay (X, Y)
\STATE Record $\leftarrow$ getClosestHit ( Ray )
\STATE
\IF {Hit}
 \IF{La surface est réfléchissante}
    \STATE Lancer le rayon réfléchi.
 \ENDIF

 \IF{La surface est transparente}
    \STATE Lancer le rayon transmis.
 \ENDIF
 \STATE
 \STATE FinalColor $\leftarrow$ Couleur ambiente + 
  Contribution Lumineuse * (Couleur de l'objet + Couleur du rayon transmis +
  Couleur du rayon réfléchi)
\ENDIF
\STATE
\RETURN FinalColor
\end{algorithmic}
\end{algorithm}


Il est impressionnant de constater la simplicité d'un algorithme pourtant si
puissant. En quelques lignes, nous venons de décrire la propagation de la
lumière aux travers d'objets transparents et/ou réfléchissant !

Nous verrons cependant que, pour rester ouvert à l'extension et garder un
couplage faible, une architecture robuste est nécessaire.
