\section{Geometry}
Ce module regroupe l'ensemble des géométries que le programme peut calculer.
Chacune de ces classes doivent respecter l'interface définit à la
\tsl{fig. \ref{fig:IGeometry}}.

\begin{figure}[h]
  \inputcode{c++}{../../code/src/Geometry/Geometry.hpp}
  \caption{Code de l'interface commune à toutes les
  géométrie \label{fig:IGeometry}}
\end{figure}

\subsection{Spécialisations}
\paragraph{[Implémentée] Sphère}
\paragraph{[Implémentée] Plane}
\paragraph{[Implémentée] Box} Cette géométrie est uniquement utilisée pour la
structure d'Octree.

\paragraph{[Implémentée] Triangle}

\paragraph{[Implémentée] Mesh} Qui est en réalité un aggrégat de triangles
stockés dans un octree.

\paragraph{[Non implémentée] Quadric} Les \tsl{quadrics} sont un ensemble de
géométrie paramétré par une équation de la forme $$xQx^T + Px^T + R = 0$$. 

Cette équation permet de décrire tout une gamme de forme allant de l'ellipse à
la sphère en passant par plusieurs types de cônes.\footnote{Pour la liste
complète, je vous invite à consulter
\url{http://en.wikipedia.org/wiki/Quadric}.}.

\subsection{Diagramme de classe}
\begin{figure}[h]
\begin{center}
  \includegraphics[width=\textwidth]{../../architecture/Geometries}
  \caption{Diagramme de classe du module Geometry\label{fig:CDGeometry}}
\end{center}
\end{figure}
