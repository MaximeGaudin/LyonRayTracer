\section{Geometry}
Ce module regroupe l'ensemble des géométries que le programme peut calculer.
Chacune de ces classes doivent respecter l'interface suivante :

\inputcode{c++}{../../code/src/Geometry/Geometry.hpp}

\subsection{Spécialisations}
\paragraph{[Implémentée] Sphère}
\paragraph{[Implémentée] Plane}
\paragraph{[Implémentée] Box} Cette géométrie est uniquement utilisée pour la
structure d'Octree.

\paragraph{[Implémentée] Triangle}

\paragraph{[Implémentée] Mesh} Qui est en réalité un agrégat de triangles
stockés dans un \tsl{octree}.

\paragraph{[Non implémentée] Quadric} Les \tsl{quadrics} sont un ensemble de
géométries paramétrés par une équation de la forme $$xQx^T + Px^T + R = 0$$. 

Cette équation permet de décrire toute une gamme de formes allant de l'ellipse
à la sphère en passant par plusieurs types de cônes.\footnote{Pour la liste
complète, je vous invite à consulter
\url{http://en.wikipedia.org/wiki/Quadric}.}.

\begin{figure}[h]
  \hbox to \textwidth
  {\hss\includegraphics[width=1.5\textwidth]{../../architecture/Geometries}\hss}
  \caption{Diagramme de classe du module Geometry\label{fig:CDGeometry}}
\end{figure}
