\section{Image}
La manipulation des images est évidemment au cœur du programme. Il doit être
capable de lire (pour les textures) et d'écrire (pour le rendu) plusieurs
formats d'image.

Pour cela, l'architecture du module est basé autours d'un \tsl{Builder} (\cf
\cite{DesignPatternsBook}, \tsl{p. 97}) dont une seule instance est autorisée
(\tsl{design pattern Singleton, p. 127}). Ainsi, il est possible de créer une
collection de classe capable de manipuler différents formats d'images. Par la
suite, le \tsl{Builder} pourra choisir la bonne implémentation des méthodes de
chargement et de sauvegarde de l'image correspondant au format demandé.

\subsection{Spécialisations}
\paragraph{[Implémentée] PNGHandler} Permet de manipuler le format PNG.

\paragraph{[Implémentée] JPGHandler} Permet de manipuler le format JPG. 

\paragraph{[Non implémentée] Autre} Il existe plusieurs dizaines de format
qu'il faudrait supporter.

\subsection{Diagramme de classe}
\begin{figure}[h]
\begin{center}
  \includegraphics[width=.4\textwidth]{../../architecture/ImageHandlers}
  \caption{Diagramme de classe du module Image\label{fig:CDImageHandlers}}
\end{center}
\end{figure}
