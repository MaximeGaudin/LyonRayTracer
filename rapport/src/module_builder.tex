\section{Builder}
Chaque objet constructible doit aussi définir un \tsl{builder}, \ie une classe
permettant à partir de la description textuelle de l'objet (\cf section
\ref{sec:scenemanager}), de construire sa représentation mémoire.

L'interface à respecter est celle présentée à la \tsl{fig.
\ref{fig:IBuilders}}.

\begin{figure}[h]
  \inputcode{c++}{../../code/src/Builders/Builder.hpp}
  \caption{Code de l'interface commune à tous les
  \tsl{builders}\label{fig:IBuilders}}
\end{figure}


\subsection{Spécialisations}
Un pour chaque objet constructible. On peut citer \ttt{MaterialBuilder},
\ttt{SphereBuilder}, \ttt{DirectionalBuilder}, \etc

\begin{figure}[h]
\begin{center}
  \vspace*{-4cm}
  \includegraphics[angle=90,width=.4\textwidth]{../../architecture/Builders}
  \caption{Diagramme de classe du module Sampler\label{fig:CDBuilders}}
\end{center}
\end{figure}
