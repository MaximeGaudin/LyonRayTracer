\section{Builder}
Chaque objet constructive doit aussi définir un \tsl{builder}, \ie une classe
permettant, à partir de la description textuelle de l'objet (\cf section
\ref{sec:scenemanager}), de construire se représentation mémoire.

L'interface à respecter est celle présentée à la \tsl{fig.
\ref{fig:IBuilders}}.

\begin{figure}[h]
  \inputcode{c++}{../../code/src/Builders/Builder.hpp}
  \caption{Code de l'interface commune à tous les
  \tsl{builders}\label{fig:IBuilders}}
\end{figure}


\subsection{Spécialisations}
\paragraph{[Implémentée] DefaultSampler} Ce sampler lance un seul rayon par pixel.

\paragraph{[Implémentée] SuperSampler} Ce sampler lance $N$ rayons par pixel où $N$ est
passé en paramètre.

\paragraph{[Non implémentée] Adaptative sampling} Il existe un sampler que je
n'ai pas eu le temps d'implémenter et qui pourtant définit la stratégie la
plus intelligente d'échantillonnage. Il s'agit de l'échantillonnage adaptatif
qui, tant que la moyenne des pixels environnant est supérieure à un certain
seuil de différence, lance des rayons selon une distribution statistique.

\subsection{Diagramme de classe}
\begin{figure}[h]
\begin{center}
  \includegraphics[width=\textwidth]{../../architecture/Builders}
  \caption{Diagramme de classe du module Sampler\label{fig:CDBuilders}}
\end{center}
\end{figure}
