\section{Description}
\textsl{LyonRayTracer}, en hommage à ma ville natale : Lyon, est un ray tracer
qui, en plus de constituer mon projet de dernière année d'ingénierie, est
développé dans l'optique :
\begin{itemize}
  \item d'être facile à prendre en main,
  \item d'être portable,
  \item d'être un vrai projet : \textsl{i.e.} documenté, mis à jour
  régulièrement, doté d'une page web, \textsl{etc.}
\end{itemize}

\textsl{LRT} est de plus développé afin de constituer une base pour un moteur plus
compliqué. Pas forcément le mien d'ailleurs. Car, développer un moteur de ray tracing demande beaucoup de
code \textsl{boilerplate} et notamment au niveau mathématique. Ici, pas de
problème car ce code est déjà là et il ne vous reste plus qu'à jouer avec le
moteur ou qui sait, \textbf{contribuer au projet !}

\subsection{Fonctionnalités}
Bien qu'encore au début de son développement, \textsl{LRT} possèdes tout de même
quelques fonctionnalités :
\begin{itemize}
  \item Lumière : Lumière directionnelle.
  \item Caméra : Caméra avec perspective.
  \item Géométrie : Sphère, Plan, Triangle, Mesh (agrégat de triangles).
  \item Importeurs : Format 3ds
  \item Exporteurs : Format PNG et JPG.
\end{itemize}

Évidemment, le pipeline de ray tracing comprend la gestion des matériaux, des
reflexions et des réfractions.
