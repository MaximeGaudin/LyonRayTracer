\section{Description}
\textsl{LyonRayTracer}, en hommage à ma ville natale : Lyon, est un ray tracer
qui, en plus de constituer mon projet de dernière année d'ingénierie, est
développé dans l'optique :
\begin{itemize}
  \item d'être facile à prendre en main
  \item d'être un vrai projet, \textsl{i.e.} documenté, mis à jour
  régulièrement, doté d'une page web, \textsl{etc.}
\end{itemize}

LRT est de plus développé dans une optique de base pour un moteur plus
compliqué. En effet, développer un moteur de ray tracing demande beaucoup de
code \tsxtsl{boilerplate} et notamment au niveau des vecteurs et des matrices.
Ce code est déjà fait ici et il ne vous reste plus qu'à jouer et qui sait,
\textbf{contribuer au projet !}

\subsection{Fonctionnalités}
Bien qu'encore au début de son développement, LRT possèdes tout de même
quelques fonctionnalités :
\begin{itemize}
  \item Lumière : Lumière directionnelle.
  \item Caméra : Caméra avec perspective.
  \item Géométrie : Sphère, Triangle, Mesh (agrégat de triangles).
  \item Importeurs : Format 3ds
  \item Exporteurs : Format PNG et JPG.
\end{itemize}
