\section{Description}
\textsl{LyonRayTracer}, en hommage à ma ville natale : Lyon, est un ray tracer
qui, en plus de constituer mon projet de dernière année d'ingénierie, est
développé dans les optiques :
\begin{itemize}
  \item d'être facile à prendre en main
  \item d'être un vrai projet, \textsl{i.e.} documenté, mis à jour
  régulièrement, doté d'une page web, \textsl{etc.}
  \item de pouvoir servir de base à des fonctionnalités plus poussées comme le
  photon mapping, l'illumination globale, les caméras avec gestion de la
  profondeur de champs, un moteur d'animation ou encore la gestion des ombres
  douces.
\end{itemize}

Bien qu'encore au début de son développement, LRT possèdes les fonctionnalités
suivantes :
\begin{itemize}
  \item Lumière : Lumière directionnelle.
  \item Caméra : Caméra avec perspective.
  \item Géométrie : Sphère.
  \item Exporteurs : Possibilité d'exporter les images au format PNG et JPG.
\end{itemize}
